\documentclass[10pt]{article} 
\usepackage[utf8]{inputenc}
\usepackage[T1]{fontenc}
\usepackage[italian]{babel}

\usepackage{geometry}
\geometry{a4paper}

\usepackage{fancyhdr}
\pagestyle{fancy}
\renewcommand{\headrulewidth}{0pt}
\lhead{}\chead{}\rhead{}
\lfoot{}\cfoot{\thepage}\rfoot{}

\usepackage{graphicx}
\usepackage{booktabs}
\usepackage{paralist}
\usepackage{verbatim}
\usepackage{subfig}

\usepackage{sectsty}

\usepackage[titles,subfigure]{tocloft}
\renewcommand{\cftsecfont}{\rmfamily\mdseries\upshape}


\title{Un titolo intelligente}
\author{Patrick Predella MAT, Federico D'Eredità 151646 }
\date{}

\begin{document}
\maketitle
\tableofcontents

\section{Introduzione}
% Breve spiegazione della finalità del progetto e metodo usato per giungere a tal fine
Finalità del progetto è il monitoraggio di misure di temperatura e pressione ambientali mediante sensoristica analogica con la possibilità di comandare attuatori analogici e/o digitali tramite programmazione di un Raspberrt Pi2+.
A questo scopo abbiamo deciso di realizzare una scheda composta dai seguenti elementi principali:
\begin{itemize}
\item Un convertitore analogico-digitale per il campionamento dei segnali analogici di temperatura e pressione affiancato da una protezione per eventuali sbalzi di tensione;
\item Un convertitore digitale-analogico per il controllo di attuatori analogici anch'esso affiancato da una protezione per eventuali sbalzi di tensione;
\item Un general purpose input/output per il controllo di attuatori digitali ed il controllo delle operazioni pre-programmate nel Raspberry;
\item Scheda PiggyBack (vedasi capitolo \ref{sec:piggy}) per l'assemblaggio della componentistica;
\item Un Raspberry Pi2+ per il controllo delle linee dati, di clock e l'interazione con il GPI/O.
\end{itemize}


\section{Componentistica}
	\subsection{Cicuiteria XXX} %non mi ricordo la sigla dei componenti che hanno i piedini

		\subsubsection{ADC - Analog-to-Digital Converter}\label{sec:adc}
		% Breve introduzione 
%		- Cos'è un ADC (genericamente) e come funziona (frequenza di campionamento, vref e breve cenno all campionamento differenziale) 
%		- L'MCP xxxx (nel dettaglio), schema, pin e loro descrizione, max ratings, PGO 
%		- Come NOI lo colleghiamo ai sensori, spiegazione Ch+ Ch- e perchè è un campionamento differenziale.
	
		\subsubsection{DAC - Digital-to-Analog Converter}\label{sec:dac}
		% Come per ADC
	
		\subsubsection{GP I/O - General Purpose Input/Output Device}\label{sec:gpio}
		% Come per ADC
	
	\subsection{Componenti Hardware}
		\subsubsection{La scheda Piggy-Back e il DCDC Converter}\label{sec:piggy}
		% Cos'è, perché la usiamo, cenni sul numero di pin sequenziali, compatibilità col Raspberry, montaggio Circuiteria
		% - Il DCDC, schema e cenni sul closed-loop feedback -> garanzia di Vdd costante
		
		\subsubsection{Il Raspberry}\label{sec:rasp}
		% Massimo 3 righe su cos'è giusto per avere una descrizione a prova di idiota
		% - Collegamento e comunicazione col Piggy-Back
		
	\subsection{Il linguaggio di programmazione I2C}\label{sec:i2c}
	% Breve cenno sull'I2C che viene usato per la comunicazione con la circuiteria

\section{Realizzazione del Progetto}

	\subsection{Protezione della componentistica}
		\subsubsection{Filtri RC}
		
		\subsubsection{Diodi Zener}
	
	\subsection{Schema elettrico d'assemblaggio}



\end{document}
